\documentclass[11pt,a4paper, uplatex]{jsarticle}
%
\usepackage{amsmath,amssymb}
\usepackage{bm}
\usepackage[dvipdfmx]{graphicx}
\usepackage{ascmac}
\usepackage{listings}
\usepackage{underscore}
\usepackage{url}
\lstset{
    frame=single,
    numbers=left,
    tabsize=2
}
%
\setlength{\textwidth}{\fullwidth}
\setlength{\textheight}{40\baselineskip}
\addtolength{\textheight}{\topskip}
\setlength{\voffset}{-0.2in}
\setlength{\topmargin}{0pt}
\setlength{\headheight}{0pt}
\setlength{\headsep}{0pt}
%
\newcommand{\divergence}{\mathrm{div}\,}  %ダイバージェンス
\newcommand{\grad}{\mathrm{grad}\,}  %グラディエント
\newcommand{\rot}{\mathrm{rot}\,}  %ローテーション
%
\title{メディアネットワーク 最終レポート}
\author{1510151  栁 裕太}
\date{\today}
\begin{document}

\maketitle

\section{トップページURL}
 \url{http://medianet.inf.uec.ac.jp/~y1510151/}

※CAUTION

学内専用。学外からはVPN経由のみアクセス可能。

\section{概要}
\subsection{テーマ・趣旨}
\subsection{閲覧者に提供するもの}
\subsection{ホームページ側のメリット}

\section{ページの内容}
\subsection{ページの内容}
\subsubsection{トップページ}
\subsubsection{博物館・美術館一覧ページ}
\subsubsection{博物館・美術館詳細ページ}
\subsubsection{レビュー(BBS)}
\subsubsection{アンケート}
\subsection{各ページの関連・遷移方法}

\section{アピール内容}
\subsection{面白いと考えた点}
\subsection{工夫した点}












\end{document}
