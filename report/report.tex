\documentclass[11pt,a4paper, uplatex]{jsarticle}
%
\usepackage{amsmath,amssymb}
\usepackage{bm}
\usepackage[dvipdfmx]{graphicx}
\usepackage{ascmac}
\usepackage{listings}
\usepackage{underscore}
\usepackage{url}
\usepackage[usenames]{color}
\definecolor{dkblue}{rgb}{0.0, 0.0, 0.55}
\definecolor{dkgreen}{rgb}{0.0, 0.2, 0.13}
\definecolor{saddlebrown}{rgb}{0.55, 0.27, 0.07}
\lstset{
    frame=single,
    numbers=left,
    tabsize=2
}
\lstset{
  language        = php,
  basicstyle      = \small\ttfamily,
  keywordstyle    = \color{dkblue},
  stringstyle     = \color{red},
  identifierstyle = \color{dkgreen},
  commentstyle    = \color{gray},
  emph            =[1]{php},
  emphstyle       =[1]\color{black},
  emph            =[2]{if,and,or,else},
  emphstyle       =[2]\color{saddlebrown}
}
%
\setlength{\textwidth}{\fullwidth}
\setlength{\textheight}{40\baselineskip}
\addtolength{\textheight}{\topskip}
\setlength{\voffset}{-0.2in}
\setlength{\topmargin}{0pt}
\setlength{\headheight}{0pt}
\setlength{\headsep}{0pt}
%
\newcommand{\divergence}{\mathrm{div}\,}  %ダイバージェンス
\newcommand{\grad}{\mathrm{grad}\,}  %グラディエント
\newcommand{\rot}{\mathrm{rot}\,}  %ローテーション
%
\title{メディアネットワーク 最終レポート}
\author{1510151  栁 裕太}
\date{\today}
\begin{document}

\maketitle

\section{トップページURL}
Museums

 \url{http://medianet.inf.uec.ac.jp/~y1510151/}

※CAUTION

学内専用。学外からはVPN経由のみアクセス可能。

\section{概要}
\subsection{テーマ・趣旨}
関東及び東京都内には、多くの博物館や美術館が点在している。
更に、一部では学生証の提示で無償あるいは割引で入館できるため、
非常に訪れやすい環境が整っている。
しかしながら、何処に何を扱う博物館があるのか、或いは何を見るべきなのかが浸透していないのか、
あまり訪れる学生が少ないのが現状である。
そこで、一都六県、特に東京都内の博物館・美術館を一覧で表示し、
公式HPへのリンクをまとめることで、どの博物館・美術館に行くかを調べやすくするWEBサイト
「Museums」を作成することにした。

\subsection{閲覧者に提供するもの\label{info}}
一都六県(主に東京都内)に点在する博物館・美術館の
\begin{itemize}
 \item 概要
 \item 外観
 \item 公式HPへのリンク
\end{itemize}
を提供する。また、詳しい情報は公式HPへのリンクによって得ることが可能。

\subsection{ホームページ側のメリット}
点在する博物館・美術館を1つにまとめることにより、簡単に行きたい所を捜すことが可能となる。
また、レビュー(BBS)により、実際に訪れた人の感想といった情報も簡単に得られ、
自らも投稿することで行きたい人・行った人によるコミュニティを形成することができる。

\section{ページの内容}

\subsection{トップページ}
所蔵している博物館・美術館の外観がページ上部にスライドショーで存在する。
また、その下にホームページの説明(概要、所蔵している施設の説明)が入る。

\subsection{博物館・美術館一覧ページ}
所蔵している博物館・美術館の外観・概要の一部(一部冒頭のみ)が入る。
概要の文章が長い場合は、後部が省略されて"read more"と続きを読むためのリンクが入る。
このリンクを選択すると詳細ページへ遷移される。

\subsection{博物館・美術館詳細ページ}
各博物館・美術館の詳細が表示される。
記述される内容は\ref{info}節の通りである。

\subsection{レビュー(BBS)}
講義内で扱ったレビューをここに適用した。

\subsection{アンケート}
アンケート内容は以下の通りである。
\begin{itemize}
 \item 名前
 \item コース
 \item 訪れたことがある博物館・美術館
 \item 今後訪れたい博物館・美術館
\end{itemize}
なお、上2つの項目は講義のままであり、
下2つの項目はチェックボックスで複数選択・無選択が可能である。

\subsection{ログインページ}
ユーザ作成機能とログイン機能を併せ持っている。
その場でユーザを作成してログインすることが可能となっている。
ログインページの内部の情報はほぼ空となっている。

\section{各ページの関連・遷移方法}
すべてのページに共通してサイドバーにホームページののロゴと各ページ(詳細ページ除く)へのリンク、
そしてツイート共有へのリンクがが掲載されている。
Twitter以外の共有リンクについてであるが、他のサービスも検討したものの、
ホームページの所在がプロキシを経由しているため、実装を断念した。

また、同時に全ページにフッターが設置されており、開発者の連絡先とホームページの概要が記載されている。

\section{アピール内容}
\subsection{面白いと考えた点}
\subsubsection{アンケートの設問}
今回、アンケートとして「訪れた」「訪れたい」博物館・美術館の調査を行った。
ここで、講義では扱っていないラジオボタン方式ではなく、チェックボックス形式を採用することで、
複数の選択を可能なものとした点が挙げられる。


\subsection{工夫した点}
\subsubsection{チェックボックスによるアンケート質問の集計方法}
先述のチェックボックスによる集計方法であるが、以下に調査・結果ページの該当する処理を記述する。
\begin{lstlisting}[caption=enquete.html: 調査ページ一部,label=enquete,language=html]
  <label for="kahaku_box_vi" >
   <input
    type="checkbox"
    id="kahaku_box_vi"
    name="visited[]"
    value="kahaku_vi"
    >
    国立科学博物館<br>
  </label>
\end{lstlisting}

調査ページから一部を抜粋した。各設問ごとに\texttt{name}タグの名前の末尾に\texttt{[]}
を配置し、各質問における解答を配列化している。

\begin{lstlisting}[caption=enq_result.php: 結果表示ページ,label=result,language=php]
  $cnt["kahaku_vi"] = 0;
  $cnt["nmwa_vi"] = 0;
  $cnt["tnm_vi"] = 0;
  $cnt["parasite_vi"] = 0;

  $cnt["kahaku_wa"] = 0;
  $cnt["nmwa_wa"] = 0;
  $cnt["tnm_wa"] = 0;
  $cnt["parasite_wa"] = 0;

  while ($csvline = fgets($fp)) {
    $data = explode(",", trim($csvline, "\n"));
    for ($i=2; $i < count($data); $i++) {
      $menu = (string)$data[$i];
      if (isset($cnt[$menu])) {
        $cnt[$menu]++;
      }
    }
  }$
\end{lstlisting}

集計ページでは、まず各項目の結果を入れる部分を初期化し、
その後結果が記録されたCSVファイルを開き、カンマで分割し、
記録された該当する変数をインクリメントしている。

\end{document}
